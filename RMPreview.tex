%\documentclass[iop,twocolumn, tighten, twocolappendix, numberedappendix]{emulateapj}
\documentclass[iop,onecolumn]{revtex4}

\usepackage[backref,breaklinks,colorlinks,citecolor=blue]{hyperref}  
\usepackage[all]{hypcap}
\renewcommand*{\backref}[1]{[#1]}
\newcommand{\x}{\sout}


\usepackage{amsmath}
\usepackage{amssymb}
\usepackage{graphicx}
\usepackage{color}
\usepackage{natbib}
\usepackage{xspace}
\usepackage{nicefrac}
\bibliographystyle{hapj}
\usepackage[dvipsnames]{xcolor}
\newcommand\myshade{85}
\usepackage[T1]{fontenc}
\usepackage{lmodern}
\usepackage{tablefootnote}
\usepackage{ulem}
 	
\colorlet{mycitecolor}{Turquoise}
\colorlet{mylinkcolor}{Turquoise}

\hypersetup{
 citecolor = mycitecolor!\myshade!black,
 linkcolor = mylinkcolor!\myshade!black,
 colorlinks = true,
}

\def\aj{Astronomical Journal}             % Astronomical Journal
\def\apj{Astrophysical Journal}                % Astrophysical Journal
\def\apjl{Astrophysical Journal}             % Astrophysical Journal, Letters
\def\pasj{PASJ}
\def\apjs{ApJS}              % Astrophysical Journal, Supplement
\def\mnras{MNRAS}            % Monthly Notices of the RAS
\def\prd{Phys.~Rev.~D}       % Physical Review D
\def\prl{Phys.~Rev.~Lett}    % Physical Review Letters
\def\cqg{Class.~Quant.~Grav.~}%Classical and Quantum Gravity
\def\araa{ARA\&A}             % Annual Review of Astron and Astrophys
\def\nat{Nature}              % Nature
\def\aap{A\&A}                % Astronomy and Astrophysics
\def\na{New Astronomy}
\def\nar{New Astronomy Reviews}
\def\pasa{Publications of the Astron.~Soc.~of Australia}
\def\aapr{Astron.~\& Astrophys.~Reviews}
\def\physrep{Physics Reports}


\newcommand{\todo}[1]{\textcolor{red}{#1}}
\newcommand{\tocheck}[1]{\textcolor{green}{Check: #1}}
\newcommand{\citeme}{{\color{magenta} CITE ME!}}
\newcommand{\ajf}[1]{\textcolor{red}{AtoI: #1}}
\newcommand{\ilya}[1]{\textcolor{magenta}{#1}}

\def\be{\begin{equation}}
\def\ee{\end{equation}}
\def\ba{\begin{eqnarray}}
\def\ea{\end{eqnarray}}
\def\rsquo{'}


%\submitted{Submit ted \today}

\begin{document}

\title{Merging stellar-mass binary black holes}
%\shorttitle{}

\def\addBham{Institute of Gravitational Wave Astronomy and School of Physics and Astronomy, University of Birmingham, Edgbaston, Birmingham B15 2TT, United Kingdom}


\author{Ilya Mandel}
\email{imandel@star.sr.bham.ac.uk}
\affiliation{\addBham}
\author{Alison Farmer}
\affiliation{kW Engineering, Oakland, California, 94612, USA}

\begin{abstract}

LIGO and Virgo detectors \ilya{have} recently directly observed gravitational waves from \ilya{the mergers of} several pairs of  stellar-mass black holes, as well as from one \ilya{merging} neutron star binary.  These observations raise the hope that compact object mergers could be used as a probe of stellar and binary evolution, and perhaps of stellar dynamics.    This informal note summarises the existing observations, describes theoretical predictions for formation channels of merging stellar-mass black-hole binaries along with their rates and observable properties, and presents some of the prospects \ilya{for} gravitational-wave astronomy.

\end{abstract}

\maketitle

\section{Introduction}

The subject of merging stellar-mass \ilya{black-hole} binaries as gravitational-wave sources is \sout{perfectly suited to an} \ilya{suitable for an}  informal colloquium\ilya{--style article} \sout{presentation}. Several formal reviews were written in the excitement following the first detection \citep{GW150914} of gravitational waves by the Laser Interferometer Gravitational-wave Observatory (LIGO) \citep{GW150914:detectors}, including the astrophysical context companion paper by the the LIGO-Virgo scientific collaboration \citep{GW150914:astro} and the excellent article by \citet{Miller:2016}. More reviews are certainly in the works. However, it is somewhat premature to formally review a field that is both rapidly evolving and very much in its infancy.  With just a few detections to date of gravitational waves from merging black holes \citep{GW150914,GW151226,BBH:O1,GW170104,GW170608,GW170814}, and one from merging neutron stars \citep{GW170817}, we are only beginning to explore the population of merging compact-object binaries.  We are not yet certain of the dominant formation channels for these sources, or even of the full range of astrophysical questions that gravitational-wave observations will answer. %It is clear, however, that there are many important and exciting questions to explore.

\ilya{What can gravitational-wave observations of merging binaries tell us about massive stellar and binary evolution that preceded the merger?  Can they constrain the amount of mass loss and expansion in massive stars?  The stability and consequences of mass transfer, including the common-envelope phase?  The amount of mass fallback and the kicks received during supernovae that give rise to neutron stars and black holes?  Does the distribution of merging compact objects contain an imprint of the star formation history in the Universe?    Can we use these mergers to probe dynamics in dense stellar environments such as globular clusters?  The answers to these important and exciting questions will in turn influence our understanding of topics as diverse as reionization and heavy-element nucleosynthesis.}

%\textcolor{red}{Can we go bigger picture here and say (briefly) why binary star evolution is interesting and/or important? A very quick summary of the Prospects section?}

In this colloquium\ilya{-style article}, we focus on the questions that strike us as being most interesting and timely at this stage of the field's development.  These will indeed mostly be questions rather than answers, and the list is undoubtedly biased by our own interests. Further, we do not attempt to systematically survey all of the relevant literature.  In other words, this is emphatically {\it not} a review. \ajf{Magritte!!!} %\textcolor{red}{Can we summarize the questions in one or two sentences? Understanding the evolution of massive stellar binaries? Something else?}


Below, we summarise the existing observations of merging binary black holes and closely related systems in section \ref{obs}, describe the plausible formation scenarios for binary black holes in \autoref{form}, discuss the predicted merger rates and merging object properties in \autoref{merge}, and speculate about the prospects for gravitational-wave astronomy in \autoref{prospect}.


 

\section{Observations}\label{obs}

\subsection{Gravitational-wave observations}


\subsubsection{The confirmed detections to date}
During its first science run, lasting from September of 2015 through January of 2016, the LIGO detector network observed three likely signals from the mergers of binary black holes.  GW150914 \citep{GW150914} and GW151226 \citep{GW151226} were confident, $> 5\sigma$ observations; LVT151012 was estimated as having a $\sim 90\%$ probability of being a gravitational-wave signal \citep{GW150914:rates,BBH:O1}, though the estimate is conservative and we shall treat it as a detection here.  During the second science run, lasting with some interruptions from November 2016 through August 2017, three further confident binary black hole detections have been announced: GW170104 \citep{GW170104},  GW170608 \citep{GW170608}, and GW170814 \citep{GW170814}, the last of them with the participation of the Virgo gravitational-wave observatory \ilya{\citep{AdvVirgo}}.   The observation of gravitational waves from the binary neutron star merger GW170817 \citep{GW170817} was followed up by an unprecedented campaign of electromagnetic observations, leading to detections of a short gamma ray burst, optical kilonova, and an afterglow spanning from radio through X-ray wavelengths \citep{GW170817:GRB,GW170817:MMA}.  We shall focus on binary black holes here.

\subsubsection{Information encoded in GW signals}
The gravitational-wave signature of a merging binary encodes the properties of the binary: the component masses and spins. Coupled with information from two or more detectors in a network, this makes it possible to infer the sky location and orientation of the source and the distance to the source \citep{Veitch:2015,GW150914:PE}.  However, some of the source parameters are strongly correlated, leading to near-degeneracies when attempting to extract them from a noisy data set.   For the heaviest black hole binaries such as GW150914, the total mass $M \equiv M_1 + M_2$ is measured relatively accurately as the ringdown, whose frequency is a function of the total mass, falls in the sensitive frequency band of the gravitational-wave detectors.  For most black holes, the chirp mass $M_c \equiv M_1^{3/5} M_2^{3/5} M^{-1/5}$ is the better measured parameter, since it determines to lowest order the rate of the frequency evolution during the inspiral phase.  The mass ratio $q\equiv M_2/M_1 \leq 1$ is often quite poorly constrained, because it enters the inspiral phasing at a higher order in the ratio of the orbital velocity to the speed of light and is partially degenerate with the black hole spins \citep[e.g.,][]{PoissonWill:1995}.  

\subsubsection{Information extracted from the observed GW signals}
We list the parameters of the black holes observed to date in table \ref{table:BHmasses}. Specific system properties are discussed below.

\begin{table}
\begin{tabular}{lcccccc}
Event  & $M_c$ [$M_\odot$]  & q & $M_1$ [$M_\odot$]  & $M_2$ [$M_\odot$]  & $\chi_\textrm{eff}$ & $d_L$ [Mpc] \\
\hline
GW150914\footnote{\citet{BBH:O1}} & $28.1^{+1.8}_{-1.5}$ & $0.81^{+0.17}_{-0.2}$ & $36.2^{+5.2}_{-3.8}$ & $29.1^{+3.7}_{-4.4}$ & $-0.06^{+0.14}_{-0.14}$ & $420^{+150}_{-180}$\\
LVT151012$^\mathrm{a}$ & $15.1^{+1.4}_{-1.1}$ &$0.57^{+0.38}_{-0.37}$ &$23^{+18}_{-6}$ &$13^{+4}_{-5}$ &$0.03^{+0.31}_{-0.20}$ &$1020^{+500}_{-490}$\\
GW151226$^\mathrm{a}$ & $8.88^{+0.33}_{-0.28}$ &$0.52^{+0.40}_{-0.29}$ &$14.2^{+8.3}_{-3.7}$ &$7.5^{+2.3}_{-2.3}$ &$0.21^{+0.20}_{-0.10}$ &$440^{+180}_{-190}$ \\
GW170104\footnote{\citet{GW170104}} & $25.1^{+2.5}_{-3.9}$ &$0.62^{+0.32}_{-0.26}$ &$31.2^{+8.4}_{-6.0}$ &$19.4^{+5.3}_{-5.9}$ &$-0.12^{+0.21}_{-0.30}$ &$880^{+450}_{-390}$\\
GW170608\footnote{\citet{GW170608}} & $7.9^{+0.2}_{-0.2}$ &$0.6^{+0.3}_{-0.4}$ &$12^{+7}_{-2}$ &$7^{+2}_{-2}$ &$0.07^{+0.23}_{-0.09}$ &$340^{+140}_{-140}$ \\
GW170814\footnote{\citet{GW170814}} & $24.1^{+1.4}_{-1.1}$ & &$30.5^{+5.7}_{-3.0}$ &$25.3^{+2.8}_{-4.2}$ &$0.06^{+0.12}_{-0.12}$ & $540^{+130}_{-210}$ \\
\hline
\end{tabular}
\caption{The parameters of merging binary black holes \ilya{inferred from gravitational-wave observations}.  The median value and the 90\% credible interval bounds are given.}\label{table:BHmasses}
\end{table}

\textbf{\ilya{Masses:}} \ilya{The individual black hole masses span a range from $\approx 7 M_\odot$ to $\approx 35 M_\odot$.  The masses appear to be roughly uniformly distributed over this range.  However, there is a significant selection bias  toward detecting more massive systems.}  Sensitivity to gravitational waves from binary mergers depends on the \ilya{system} mass.  The \ilya{gravitational-wave amplitude, and hence the maximum (horizon)} distance at which a source is detectable, scales as $M_c^{5/6}$ for inspiral-dominated signals, so the surveyed volume scales as $M_c^{2.5}$.  While \ilya{existing} observations are not sufficient to \ilya{accurately} determine the mass distribution of merging binary black holes, the observed broad mass distribution, deconvolved with the mass-dependent selection effects, could be consistent with an intrinsic mass distribution of $M^{-2.5}$ \ilya{\citep{BBH:O1}}, reminiscent of the stellar initial mass function \citep{Salpeter:1955}.  The uncertainty in the mass distribution is relayed into the uncertainty on inferred merger rates; these later are consistent with a broad range spanning $\sim 10$ -- $200$ Gpc$^{-3}$ yr$^{-1}$ \ilya{\citep{GW170104}}.  

\textbf{Mass Ratios:} While all binary black holes observed today are consistent with having equal-mass components, mass ratios could be as extreme as $4:1$ in some cases.  

\textbf{Black Hole Spins:} Individual black hole spins are difficult to measure precisely with gravitational waves.  However, we do know that none of the observed binaries could have large spins aligned with the orbital angular momentum; in fact the averaged dimensionless spin along the direction of the orbital angular momentum $\chi_\textrm{eff}$ is below $0.4$ for all observed events, and consistent with zero for all but one of them.  Interestingly, it is not consistent with zero for GW151226, the second lightest binary black hole detected, indicating that at least one of the components for that event must have been a rotating Kerr black hole.  Spin projections in the orbital plane are very poorly constrained for all events.

\textbf{Distances:} All signals came from distances of around 500 Mpc to 1 Gpc (redshifts $z\sim 0.1$ -- $0.2$).  With just two operational detectors, LIGO Hanford and LIGO Livingston, for all events prior to GW170814, sources could only be localised to 90\% credible regions spanning hundreds to more than a thousand square degrees on the sky.  The participation of the Virgo instrument in the observation of GW170608 reduced the sky region to only 60 deg$^2$ \ilya{\citep{GW170608}}.  Nonetheless, associations with specific host galaxies are impossible for all binary black hole mergers.  

\textbf{Testing General Relativity:} All gravitational-wave signals observed to date are consistent with gravitational waves expected within the general theory of relativity, providing a stringent test of this theory in the dynamical, strong-field regime \citep{GW150914:GR}.

\subsection{Electromagnetic observations}

%\subsubsection{Electromagnetic observations as complementary probes}

\ilya{While recent gravitational-wave observations have invigorated the field of massive binary evolution, there is already a wealth of electromagnetic observations of various stages along the possible paths to binary black hole formation, which are discussed in \autoref{form}.  In this section we describe some of the key electromagnetic observations --- together with the GW observations of merging black holes --- shed light on compact object binary formation and evolution preceding merger.}

\ilya{Electromagnetic observations of the evolution of massive stellar binaries include: 
\begin{itemize}
\item observations of the initial mass and period distributions of binary stars at formation \citep[e.g.,][]{Sana:2012}; 
\item observations of luminous red novae, which may be associated with common envelope events \citep[e.g.,][]{Ivanova:2013LRN};
\item Galactic binary radio pulsars \citep[e.g.,][]{Tauris:2017};
\item short gamma-ray bursts \citep[e.g..][]{Berger:2014};
\item supernovae and long gamma-ray bursts;
\item X-ray binaries.
\end{itemize}
Here, we focus on X-ray binaries consisting of a star transferring mass onto a black hole companion, leading to the emission of X-ray radiation from the accretion disk surrounding the black hole, since this is the type of binary star with the closest connection to merging binary black holes.  In fact, a small minority of black-hole X-ray binaries will ultimately evolve into merging black holes \citep{CygnusX3:2012}.}

\subsubsection{X-Ray Binaries}

\ilya{X-ray binaries with accreting black holes can be divided into two categories based on the impact of the tidal pull of the black hole on the companion.  If the tidal field of the black hole is disrupting the companion, forcing mass to stream onto the black hole in a process known as Roche-lobe overflow, the systems is known as a low-mass black-hole X-ray binary.  If the tides are less extreme and the black hole only accretes a fraction of the mass driven off the stellar surface (stellar wind), the system is known as a high-mass X-ray binary.}

\textbf{Black hole masses:} 
Black-hole X-ray binaries with dynamical mass measurements provide the only other secure measurements of black hole masses in addition to the gravitational-wave observations described above.  \citet{Ozel:2010} and \citet{Farr:2011} summarize the mass distribution of black-hole X-ray binaries; they find a distribution which ranges from 4 or 5 solar masses \ilya{for the lightest low-mass X-ray binaries} (with a possible mass gap between neutron star and black hole masses) to $\sim 15\, M_\odot$ or more \ilya{for the heaviest high-mass X-ray binaries}.  However, given the doubts subsequently placed on the radial velocity measurements in high-mass X-ray binaries such as IC10 X-1 \citep{Laycock:2015}, there are no known black holes with dynamically confirmed masses guaranteed to exceed $15\, M_\odot$.  The masses of black holes in X-ray binaries are sketched in figure \ref{fig:BHmasses} along with the masses of gravitational-wave sources.  This figure does not include speculative but potentially very exciting intermediate-mass black holes \citep{MillerColbert:2004,Pasham:2014}; if such few-hundred solar-mass black holes generically exist in globular clusters, they will be detectable as gravitational-wave sources \citep[e.g.,][]{Mandel:2008}.
%\todo{Add GC X-ray binary observations} 

\begin{figure}
	\centering
	%\includegraphics[trim={1.3cm 7.0cm 0.0cm 7.9cm},clip,scale=0.45]{evolvedPvseWithOrbit.pdf}
	\includegraphics[width=0.99\textwidth]{Graveyard.png}%{BHmass.png}
	\caption{\label{fig:BHmasses}  The masses (in solar masses, vertical axis) of black holes observed as gravitational-wave sources (blue), as well as masses of black holes in X-ray binaries (purple).  Galactic neutron stars observed as radio pulsars (yellow) and the merging double neutron star GW170817 (orange) appear at the bottom of the figure.  Placement on the horizontal axis is arbitrary.  Error bars are indicative of the measurement uncertainty.  Figure courtesy of Frank Elavsky, Northwestern University and LIGO-Virgo collaborations.}
\end{figure}


\textbf{Black Hole Spin Magnitudes:} Black hole X-ray binaries also provide an opportunity for measuring black hole spins \citep[see][for a recent review]{MillerMiller:2015}.  Continuum fitting of the X-ray flux from the disk and iron K-$\alpha$ line fits to the disk reflection profile can both be used to infer the location of the inner edge of the disk, assumed to correspond to the radius of the innermost stable circular orbit, which is a sensitive function of black hole spin.  Quasi-periodic oscillations also have the potential to provide spin measurements.  Unfortunately, the underlying physical mechanisms are not fully understood at present, and the models can suffer from significant systematics.  For example, \ilya{out of six} systems for which both disk continuum and disk reflection spin measurements are available, the two methods are inconsistent at the $3$ or $5$ sigma level \ilya{for two systems}, and for one system the statistical uncertainty is so large as to span nearly the full allowed range from $0$ to $1$.  However, for the remaining \ilya{three} systems with both measurements available, both methods yield spins $\chi \gtrsim 0.9$.  

\textbf{Inconsistency with Spin Observations from GW Sources?} Most relevantly, these high-spin observations include two high-mass X-ray binaries: LMC X-1 and Cygnus X-1.  This is significant, because unlike long-lived low-mass X-ray binaries, whose spin magnitudes could be altered by accretion from the companion, especially if they started out with intermediate-mass companions \citep{Podsiadlowski:2003,Fragos:2015}, high-mass X-ray binaries are too short-lived to enable significant spin changes due to accretion \citep{KingKolb:1999}.  %Moreover, such high-mass X-ray binaries may be direct progenitors of merging binary black holes \citep{Bulik:2008,CygnusX3:2012}; in fact, 
Isolated binaries that form binary black holes must go through the high-mass X-ray binary phase during their evolution.  Thus, if the high spin magnitude measurements in black-hole X-ray binaries are to be believed, at least some stellar-mass black holes in systems similar to those that could go on to become merging black holes should have high spins.  However, there may be significant differences in the evolutionary channels and environments of the locally observed high-mass X-ray binaries and the binary black holes \ilya{observed in gravitational waves} \citep{HotokezakaPiran:2017}.  Moreover, the spin magnitude measurements \ilya{in X-ray binaries} could be very sensitive to the modelling assumptions, and the systematic errors may dominate the claimed statistical ones \citep[e.g.,][]{Basak:2017,Kawano:2017}.

\textbf{Black Hole Spin Directions:} We know even less about the spin directions than the spin magnitudes.  Although black-hole spin-orbit alignment is assumed in continuum flux measurements \citep{MillerMiller:2015}, some black-hole X-ray binaries, including GRO J1655-40 \citep{Martin:2008}, 4U 1543-47 \citep{MorningstarMiller:2014} and particularly V4641 Sgr \citep{Orosz:2001,Martin:2008b} appear to indicate that the microquasar jet, presumably aligned with the BH spin axis, is misaligned with the orbit.  Moreover, initial stellar spins in massive binaries have been observed to be misaligned \citep[e.g.,][]{Albrecht:2009,Albrecht:2014}, though there are opportunities for realignment during binary evolution, as described below.

\section{Formation scenarios}\label{form}

When the detection of GW150914 was first announced, many were surprised that it was a binary black hole rather than a binary neutron star (\ilya{because there was observational evidence for merging double neutron stars in the Galaxy starting with the Hulse-Taylor binary pulsar \citep{HulseTaylor:1975}, but no direct evidence for merging binary black holes}), and even more surprised by the high black hole masses (in excess of the observed black hole masses in X-ray binaries). Was this surprise justified? And should we be \ilya{perplexed by} the spin or mass ratio measurements for the \ilya{gravitational-wave} sources? Before answering these questions, we should first discuss how these merging compact binaries come into existence. In fact, this is one of the key outstanding questions that gravitational wave observations will help us to answer. Below, we explain why -- on the face of it -- it is rather \ilya{startling} that compact massive binaries exist at all. We then describe the leading candidate formation channels, and in the next section we \ilya{discuss} the ways in which gravitational-wave observations might be used to distinguish them.


\subsection{The Separation Question}


\textbf{Only very tight binaries can merge via gravitational waves.} Gravitational-wave emission is a very strong function of separation. During a compact binary merger, the luminosity in gravitational waves is a few thousandths of the Planck luminosity, $c^5/G$ \citep[e.g.,][]{Cardoso:2018}; at nearly $10^{57}$ ergs per second, such mergers ``outshine'' all the stars in the visible Universe combined. But because gravitational-wave luminosity is inversely proportional to the fifth power of the binary separation \citep{Peters:1964}, widely separated binaries lose energy very slowly and undergo only negligible inspiral over billions of years. Only very close binaries can be brought to merger by gravitational waves within the age of the universe; \autoref{fig:periapsis} shows the maximum initial separation for an equal-mass binary to merge on this timescale. For the two thirty-solar mass black holes responsible for GW150914, the initial separation must have been less than 50 solar radii -- just a quarter of the distance from the Earth to the Sun -- if the merger was driven by gravitational-wave emission alone.

\textbf{But the black holes' parent stars can't get so close.} Stars expand as they evolve. Even our Sun will reach \ilya{roughly} an astronomical unit in size during its giant phase; the stars (above 20 solar masses at birth) that leave behind black holes may reach thousands of solar radii at their maximal extent. Maximum stellar radius as a function of initial mass is plotted in \autoref{fig:Rmax}. We appear to have a problem. If the parent stars begin life at separations from which gravitational waves could bring their remnants together, the stars will expand to sizes larger than their separation as they evolve, and will therefore merge long before they collapse into black holes. If they start sufficiently far apart to avoid merger before collapse, their remnant binaries will take many millions of times the age of the universe to merge. In either case, no gravitational-wave sources would exist today. 

\ilya{\textbf{The problem is already there at birth.} In fact, for reasonable models of wind-driven mass loss and mass loss during supernova, even initial stellar radii at the start of the main sequence are too large to fit into a binary that could merge within the age of the Universe just through gravitational-wave emission.  The black 'Roche lobe' curve in \autoref{fig:Rmax} shows the maximum size that a star could have in order to fit into a circular, merging equal-mass binary (see \autoref{fig:periapsis}).  This is smaller than the zero-age main sequence radius, so the problem is there even without accounting for stellar radial expansion or binary widening through mass loss, which would exacerbate the problem.}

%Because the gravitational-wave merger timescale is inversely proportional to the cube of the masses, double neutron stars must be separated by less than 5 solar radii to merge within the age of the Universe. The stars (\ilya{approximately} 8 to 20 solar masses at birth) that produce neutron stars also expand greatly during their evolution. But even \textit{at birth}, the neutron stars' progenitors could not fit into a binary of the required size.  
 
\begin{figure}
	\centering
	\includegraphics[width=0.4\textwidth]{M-rp-log.png}
	\caption{Maximum initial periapsis separation that a binary black hole with equal-mass components (as given on the ordinate) can have while still merging within the age of the Universe through gravitational-wave emission, \ilya{for four different choices of initial eccentricity: $e=0$, 0.9, 0.99, and 0.999 from the top down}.\label{fig:periapsis}}
\end{figure}
	
\begin{figure}
	\centering	
	\includegraphics[width=0.9\textwidth]{StellarRadiusZsolarRoche.png}
\caption{Maximal stellar extent at solar metallicity during various phases of stellar evolution for a non-rotating star with a given initial mass; based on the COMPAS implementation of single stellar evolution models of \citet{Hurley:2000}.   \ilya{The star expands from its zero-age main sequence (ZAMS) size onwards through the main sequence (MS), Hertzsprung gap (HG), core Helium burning (CHeB) stages, and the asymptotic giant branch (AGB) for lower-mass stars.  The Roche lobe radius (maximal stellar size beyond which a star would engage in mass transfer \citep{Eggleton:1983}) is plotted for an equal mass circular binary which would have the maximum separation to allow a merger within the age of the Universe (see \autoref{fig:periapsis}) assuming initial mass to compact object mass conversion as in \autoref{fig:BHremnant}.}\label{fig:Rmax} }
\end{figure}

%\todo{Update left panel of plot to show periapsis on horizontal axis.  Take max radial extent plot from Simon?}

\textbf{So then why do the \ilya{gravitational-wave} sources exist at all?} Given the above, one might conclude that gravitational-wave driven compact binary mergers do not exist. Yet their existence and detection were expected \ilya{\citep{ratesdoc}}. \ilya{In fact, \citet{Dyson:1962} conjectured about the existence of merging neutron star binaries even before the first neutron star was observed; \citet{Tutukov:1973} argued that binary compact objects must naturally (albeit  rarely, and at wide separations in their model) form as a result of massive binary evolution; \citet{vdHDeLoore:1973} argued that tight high mass X-ray binaries -- progenitors of compact object binaries -- must also form; and the Hulse--Taylor binary pulsar demonstrated the existence of binary compact objects that would merge within the age of the Universe.}    Three leading formation scenarios for merging compact-object binaries have been proposed: (i) finely tuned binary evolution that brings the stars closer only after their major expansion phases have passed; (ii) finely tuned stellar evolution that prevents the parent stars from expanding at all; and (iii) assembly of a close binary from black holes that formed from isolated stars. Contrived as these scenarios sound, all three might plausibly contribute to the production of binary black hole mergers. We explore these three scenarios in more detail below. (We won't discuss some of the more exotic proposed mechanisms, such as the fragmentation of a single stellar core into two black holes \citep{Loeb:2016} -- \citet{Woosley:2016,Dai:2017} discuss the problems with this picture -- or non-astrophysical formation scenarios such as primordial black holes of cosmological origin \citep[e.g.,][]{Bird:2016}.)

\subsection{The Candidate Formation Scenarios}
\subsubsection{Bring the stars together after they have expanded: classical isolated binary evolution via the common-envelope phase}

The first possible channel is perhaps the most studied one, \ilya{placing compact objects in the same framework as other very close binaries, such as cataclysmic variables in which at least one of the stars should have extended beyond the binary's current orbital separation during an earlier stage of its evolution \citep[e.g.,][]{Paczynski:1976}}.  In this scenario the two stars are born in a relatively wide binary, allowing them space to expand.  However, at a critical moment in its evolution, the binary is tightened by a factor of two or more orders of magnitude through dynamically unstable mass transfer, known as a common envelope phase. The resulting tight binary may then be close enough to merge through gravitational-wave emission.  \ilya{\citet{SmarrBlandford:1976} may have been the first to explicitly point to this channel in the context of compact object binary formation when analyzing the evolutionary history of the Hulse--Taylor binary pulsar.} The channel has been studied at length over the past 40 years, with significant contributions from \citet{TutukovYungelson:1993,Lipunov:1997,BetheBrown:1998,Nelemans:2003,VossTauris:2003,Pfahl:2005,Dewi:2006,Kalogera:2007,OShaughnessy:2008,Dominik:2012,Belczynski:2016,EldridgeStanway:2016} and many others. Rather than summarizing all of the steps and challenges in our understanding of massive binary evolution (see the papers above and the review by \citet{PostnovYungelson:2014} for details), we provide a schematic outline of an evolutionary scenario that leads to the formation of a GW150914-like merging system.

\begin{figure}
	\centering
	\includegraphics[width=0.45\textwidth]{channel1.png}
	\caption{\label{fig:isol_binary} A sketch of merging black hole binary formation \ilya{through isolated binary evolution via the common phase.}}
\end{figure}

The evolution of this system is sketched out in a van den Heuvel -- style diagram in \autoref{fig:isol_binary}\ilya{, and the steps below follow the panels in that figure}:
\begin{itemize}  
\item{a.} Two massive stars of perhaps $100$ and $75$ solar masses are born in a low-metallicity environment ($\sim 5\%$ of solar metallicity)  binary at a separation of $\sim 10$ AU.  
\item{b.} The more massive primary reaches the end of its main sequence first.  At this stage, it has completed fusing hydrogen into helium in its core, and with the loss of energy input, the core begins to contract.  The \ilya{associated} release of gravitational binding energy and the \ilya{eventual} onset of hydrogen shell burning cause the hydrogen-rich envelope to expand.  \ilya{For a sufficiently close binary, the primary expands past the equipotential surface known as the Roche Lobe}, and beginning to transfer mass onto the secondary.  The mass transfer proceeds on the thermal timescale of the primary donor and could be significantly non-conservative, as the less evolved secondary, with its correspondingly longer thermal timescale, is unable to accept mass at the rate at which it is being donated.  The loss of mass from the binary further widens the system, to perhaps $\sim 20$ AU.  
\item{c.} The primary loses its entire envelope, leaving behind a naked helium-burning star -- a Wolf-Rayet star.  
\item{d.} Following wind-driven mass loss, which further widens the system, the primary collapses into a black hole; here, this collapse is assumed to be complete, without an associated natal kick.  
\item{e.} When, a few hundred thousand years later, the secondary reaches the end of its main sequence, the process repeats in reverse: the secondary expands until it commences mass transfer onto the primary.  By this time, the primary has lost around two thirds of its initial mass through a combination of envelope stripping, winds, and possible mass loss during a supernova.  Meanwhile, this mass transfer would need to be almost wholly non-conservative if the accretion onto the black hole obeys the Eddington limit, \ilya{which corresponds to an equilibrium between gravity and the pressure on infalling material of the radiation released during accretion.}  Consequently, mass transfer would lead to a rapid hardening of the binary at a rate that is faster than the reduction in the size of the secondary donor on mass loss.  As a result, the more mass it donates, the more the donor overflows its Roche lobe.  
\item{f.} This runaway process of dynamically unstable mass transfer leads to the formation of a common envelope of gas around the binary out of the donor's envelope; see \citet{Ivanova:2013} for a review.  The drag force on the black hole from the envelope leads to rapid spiral-in.   The dissipated orbital energy is deposited in the envelope, and may ultimately lead to the expulsion of the envelope.  
\item{g.} The orbital energy is decreased by an amount necessary to unbind the envelope, and the resulting black hole -- Wolf-Rayet binary has a separation of only $\sim 35 R_\odot$ in this example.  
\item{h.} Following further wind-driven mass loss from the secondary and its collapse into a black hole, the black hole binary is formed.  While this entire process takes only a few million years \ilya{from the formation of a stellar binary to the formation of a binary black hole}, the subsequent inspiral through gravitational-wave emission will last for around 10 billion years before merger.
\end{itemize}

Of course, this relatively simple picture holds many uncertainties: the rate of mass loss through winds, particularly during specific stellar evolutionary phases such as the luminous blue variable phase \citep{Mennekens:2014} and its dependence on metallicity; \ilya{the fraction of the donated mass that is added to the accretor during stable mass transfer and the specific angular momentum of the mass that is removed from the binary}; the response of a star to mass loss and the onset of a common envelope phase \citep{Pavlovskii:2017}; common-envelope survival and the amount of binary hardening associated with the envelope ejection; supernova fallback and natal kicks for black holes \citep[e.g.,][]{Repetto:2012,Mandel:2015kicks}; the possible contribution of double-core common envelopes when unstable mass transfer is initiated between two evolved stars; and the role of dynamically stable non-conservative mass transfer onto a lower-mass compact donor in achieving sufficient binary hardening  \citep{vandenHeuvel:2017,Neijssel:2018}.  We will discuss the possibilities of addressing some of these with future gravitational-wave observations in \autoref{prospect}.

\subsubsection{Abracadabra, thou shalt not expand: chemically homogeneous evolution}

\begin{figure}
	\centering
	\includegraphics[width=0.4\textwidth]{channel2.png}
	\caption{\label{fig:chem_homog} A sketch of merging black hole binary formation via the chemically homogeneous evolution channel.}
\end{figure}

What if some types of massive stars did not expand? A massive binary could then start out at an orbital period of a couple of days, close enough that if the stars produced black holes in situ they would merge within the age of the Universe through gravitational-wave emission. Without an expansion phase, the parent stars would be in no danger of merger: no fine-tuning of binary evolution would then be required to construct a plausible formation scenario for tight black-hole binaries. This scenario does, however, require \ilya{some unproven assumptions regarding} stellar evolution. But it may be the case that high-mass, low-metallicity stars in close binaries behave exactly as needed for this scenario to work.

Binary companions raise tides on each other, much like the Moon's tides on Earth.  If a binary is tight enough that each star fills a significant fraction of its Roche lobe, tidal energy dissipation is rapid and proceeds until the stars are tidally locked, i.e. the rotation periods of the stars are synchronized to the orbital period of the binary. This also means that the stars are rotating at a few tens of percent of their break-up velocities.  Such rapidly rotating stars will develop significant temperature gradients between the poles and the equator, which may lead to efficient large-scale meridional circulation within each star \citep{Eddington:1925,Sweet:1950}.  \citet{EndalSofia:1978} and subsequent studies \citep[e.g.,][]{Heger:2000,MaederMeynet:2000,Yoon:2006} explored the internal shears and their impact on the mixing of chemical species within the star.  Although quantitative predictions differ, it appears that rapidly rotating massive stars may efficiently transport hydrogen into the core and helium out into the envelope until nearly all of the hydrogen in the star is fused into helium.  Then, at the end of the main sequence, the star behaves essentially as a Wolf-Rayet naked helium star, contracting rather than expanding. As long as the metallicity is sufficiently low that \ilya{the wind-driven mass loss does not} significantly widen the binary -- which would lead to the loss of co-rotation and chemically homogeneous evolution \citep{deMink:2009} -- the binary can avoid mass transfer.  \citet{MandeldeMink:2016,deMinkMandel:2016}, and \citet{Marchant:2016} explored such binaries and concluded that they could present a viable channel for forming the most massive \ilya{gravitational-wave} sources (such as GW150914), though not the lowest-mass systems.  \ilya{This evolutionary pathway is sketched out in \autoref{fig:chem_homog}.}

\subsubsection{The black hole matchmaking club: dynamical formation in dense stellar environments}

\begin{figure}
	\centering
	\includegraphics[width=0.4\textwidth]{channel3.png}
	\caption{\label{fig:dynamical} A sketch of merging black hole binary formation via the dynamical evolution channel.}
\end{figure}

The last possibility we will describe is that the merging black holes may not have formed in a binary at all.  Instead they formed from the collapse of independent massive stars, but were then introduced to each other by a matchmaker, or rather a whole array of matchmakers.   \ilya{This evolution is illustrated in \autoref{fig:dynamical}:}  
\begin{itemize}
\item{a.} Suppose the two black holes formed in a dense stellar environment such as a globular cluster.  
\item{b.} As the most massive objects in the cluster, they naturally sink toward the cluster center as energy is equipartitioned throughout the cluster in the process known as mass segregation.  Once there, they may form binaries through three-body interactions, or by substituting into existing binaries: in a $2+1$ interaction, the lightest object is usually ejected in favor of the two heavier objects forming a binary.  
\item{c,d.}Subsequent interactions with other objects in the cluster -- the matchmakers -- will gradually tighten the black-hole binary: the interlopers are likely to leave with  slightly higher velocities than they arrived with, each time removing energy from the binary.   
\item{e.} If the density of objects is high enough to ensure a suitable rate of interactions, the binary will be hardened until it is compact enough to merge through gravitational-wave emission, provided it does not get kicked out of the cluster through a recoil kick from the last interaction -- and even then, it may still go on to merge outside of the cluster.  
\end{itemize}

The prospects for these dynamically formed merging binary black holes in globular clusters \ilya{and young massive stellar clusters} have been explored through a series of analytical estimates and numerical experiments by \citet{Sigurdsson:1993,Kulkarni:1993,PZwart:2000,OLeary:2006,Banerjee:2010,Downing:2011,Morscher:2015,Mapelli:2016,Rodriguez:2016} and others.  \citet{OLeary:2008} and \citet{MillerLauburg:2008} have argued that dynamical formation could also take place in galactic nuclei with and without a massive black hole, respectively (but cf.~\citet{Tsang:2013} for the former).  \citet{Bartos:2016} and \citet{Stone:2016} have described the role that an accretion disk in an active galactic nucleus could play in enhancing the rate of binary black hole mergers.

Dynamical formation and isolated binary evolution mechanisms are of course not mutually exclusive; for example, the inner binaries in hierarchical triple systems may evolve as isolated binaries, but may then be brought together more efficiently through Lidov-Kozai oscillations \citep{Lidov:1962,Kozai:1962}; see, e.g., \citet{PeretsKratter:2012,Belczynski:2014VMS}.


\section{Expected properties of merging systems}\label{merge}

In this section we describe the expected hallmarks of each formation scenario in the properties of observed binary black hole mergers. These properties include black hole masses, mass ratios, and spins, orbital eccentricities, formation environments, and merger rates.

Other than the LIGO and Virgo sources, there are no known direct observations of binary black holes (except for a very speculative potential observation through microlensing \citep{Dong:2007}). As the population of gravitational wave detections grows, so will our ability to infer merger rates and typical system properties from observed data; see the next section for a discussion of the current state of these efforts. However, our current understanding of the expected merger rates and properties of binary black holes rests largely on modelling. 

For the isolated binary evolution channel (\autoref{form}) this modeling typically uses a population synthesis approach, i.e. forward modelling of large populations of stellar binaries distributed according to observed initial mass and separation distributions, some (small) fraction of which end up as merging binary black holes. \ajf{This next sentence is a bit confusing here: move or remove? Population synthesis techniques have also been applied to specific observed systems composed of a black hole and a Wolf-Rayet star, such as Cygnus X-1 \citep{Bulik:2008} and Cygnus X-3 \citep{CygnusX3:2012}, but whose future evolution is still uncertain.}   The merger rate predictions from population synthesis models produced prior to 2010 are summarized by \citet{ratesdoc}.


\subsection{Merger rates}

 Here, rather than delving into the details of binary evolution, we will outline high-level estimates of merger rates for each of the three candidate formation channels. The value of the back-of-the-envelope merger rate analysis below is not so much in matching the rate inferred from gravitational-wave observations a posteriori, as in setting the stage for some of the predictions of the properties of merging binaries (see below) and the discussion of evolutionary uncertainties (see \autoref{prospect}). \ajf{But would be nice to say something in this section about how these derived merger rates compare to observations -- do all three channels produce rates to OOM similar to the observed rate?}

\subsubsection{Isolated binary evolution channel}
\ajf{Need to do the same thing, at least briefly, for the other 2 channels.}

In order to survive an evolutionary pathway such as the one depicted in \autoref{fig:isol_binary}, a binary must (i) have the right masses to form two black holes; (ii) have the right separation to avoid a premature merger, yet be close enough to interact; (iii) avoid disruption by supernova kicks; (iv) engage in and survive a common envelope phase; and (v) end up sufficiently compact at binary black hole formation to merge within the age of the Universe. We describe below the development of a ``Drake equation'' for the probability that a stellar binary will end its life as a merging binary black hole, addressing each of these five factors in turn.

\begin{enumerate}
	\item[(i)] The minimal initial stellar mass for forming a black hole is likely around 20 solar masses, so $f_\textrm{primary} \approx 0.1\%$ of all stars drawn from the \citet{Kroupa:2002} initial mass function will form black holes. Although this can be shifted somewhat by binary interactions, for the purpose of this back-of-the-envelope calculation, we will use this approximation. Meanwhile, assuming that the mass ratio between the primary and the secondary is roughly uniformly distributed \citep{Sana:2012}, typically half of the secondaries will fall into the mass range of interest, $f_\textrm{secondary} \approx 0.5$.  

\item[(ii)] Initial binary  separations are observed to be distributed uniformly in the logarithm, $p(a) \propto 1/a$ \citep{Opik:1924}.  The first episode of mass transfer from the donor should typically be case B mass transfer (i.e., the donor should evolve beyond core hydrogen burning, but not yet beyond core helium burning); stars expand by several orders of magnitude during this phase, which means that the range of initial separations occupies a significant logarithmic fraction of the total initial range of separations, so $f_\textrm{init sep} \approx 0.5$.  

\item[(iii)]  We assume that black holes receive low natal kicks and do not lose significant amounts of mass, so that supernovae do not significantly affect binary parameters, i.e. $f_\textrm{survive SN1} \approx f_\textrm{survive SN2} \approx 1$.  

\item[(iv)] The fraction of binaries that initiate and survive the common envelope phase during mass transfer from the secondary to the primary is perhaps the least certain.  Typically, it is easier \ajf{is there a better word to use than easier here?} to initiate a common envelope if the mass ratio of the donor to the accretor is greater, and when the donor begins mass transfer earlier in its post-main-sequence evolution.  The accreting black hole in this case may only have a third of the initial mass of the primary, with the rest lost with the envelope and through winds, while the secondary donor may have increased its mass during the first mass transfer phase; even so, unless the secondary was initially close to the primary in mass, the mass ratio is unlikely to exceed $3:1$, which is probably close to the threshold for initiating a common envelope phase.  At the same time, while an early onset of mass transfer means that the less evolved donor does not shrink as much in response to mass transfer (making mass transfer stable), it cannot have an envelope that is too tightly bound, otherwise the orbital energy is insufficient to eject it.  Together, these constraints reduce the fraction of successful common envelope initiations and ejections to $f_\textrm{CE} \approx 0.1$.  

\item[(v)]  Finally, there is the question of final separation at second black hole formation, as only systems with separations smaller than those in \autoref{fig:periapsis} will merge in the age of the Universe.  The final separation after the common envelope ejection is set by the binding energy of the envelope at the time when the secondary initiates unstable mass transfer, which in turn depends on the binary separation at that time.  Therefore, very crudely, the flat-in-the-log distribution of initial binary separations persists to the final binary separation.  Since the delay time between formation and merger $\tau_\textrm{GW} \propto a^4$ is a power law in the separation, the delay time distribution also follows a flat-in-the-log distribution $p(\tau) \propto 1/\tau$.  Among the binaries that survive a common envelope the closest will be those whose separations just barely encompass a Wolf-Rayet star, i.e., with separations of a few solar radii (and merger times of a few million years).  The logarithmic distribution ensures that tens of percent of binaries that survive a common envelope will merge in the age of the Universe, $f_\textrm{merge} \approx 0.2$.
\end{enumerate}

Thus, our Drake equation is as follows:
\begin{eqnarray}
f_\textrm{BBH} &=& f_\textrm{primary} \times f_\textrm{secondary} \times f_\textrm{init sep} \times f_\textrm{survive SN1} \times f_\textrm{CE} \times f_\textrm{survive SN2} \times f_\textrm{merge} \nonumber \\
 & \sim & 0.001 \times 0.5 \times 0.5 \times 1 \times 0.1 \times 1 \times 0.4 = 5 \times 10^{-6}.
\end{eqnarray}
Merging binary black holes are rare outcomes indeed!  [Merging binary neutron stars are similarly rare: while stars with initial masses sufficient to form a neutron star, between roughly 8 and 20 solar masses, are more common than the heavier stars necessary to form black holes, neutron star natal kicks and mass loss during supernovae are more likely to disrupt binaries; the expected yields are comparable.]

The star formation rate in the local Universe is $\sim 0.01 M_\odot$ Mpc$^{-3}$ yr$^{-1}$; for an average binary mass of $\sim M_\odot$, the yield $f_\textrm{BBH} \sim 5 \times 10^{-6}$ corresponds to a binary black hole merger rate of $\sim 100$ Gpc$^{-3}$ yr$^{-1}$, or 10 per Myr for a Milky-Way equivalent galaxy with a space density of $0.01$ Mpc$^{-3}$.  Of course, the actual calculation of the binary merger rate requires significantly more care to account for the time-varying star formation history convolved with the time delay distribution.  \ajf{We've moved from describing e.g. solar radii in words to using the symbols for everything... need to make consistent.}

\subsection{System Properties}
\subsubsection{Masses}
Once formed by stellar collapse or supernova, the black holes in the binaries of interest will have roughly constant masses. The first black hole to form is unlikely to accrete a significant amount of mass from its companion: doubling its mass at the Eddington limit would take more than 100 million years; the massive companion will only survive for a small fraction of that time. After formation of the second compact object, there is very little further accretion in the system. The mass of each black hole is thus almost equal to its mass at formation. 

The classical isolated binary evolution channel discussed in detail above can produce binary black holes with a broad range of masses, matching all observations to date \citep[e.g.,][]{Stevenson:2017} as well as GW170817 and the observed Galactic double neutron stars \citep[e.g.,][]{Kruckow:2018,VignaGomez:2018}.  In fact, contrary to mistaken lore, the high masses of the first observed black holes were not entirely surprising: such systems were predicted to arise in low-metallicity environments and make a significant contribution to the detected population because more massive binaries emit more energy in gravitational waves \citep{Dominik:2014}; the effect of metallicity is discussed further in \autoref{environ} below. 

 Dynamical formation in globular clusters can also lead to a range of masses, but may favor more massive binaries: lighter black holes would be ejected by heavier ones in three-body interactions \citep{Rodriguez:2015}. The mass distribution is sensitive to natal kicks, which may eject black holes at formation \citep{Zevin:2017}.  \citet{Chatterjee:2017} argue that dynamical formation could explain all black hole binary merger observations to date, with old, metal-poor clusters preferentially contributing more massive black hole binaries and younger, metal-rich clusters contributing lighter binaries.
 
  Chemically homogeneous evolution is only possible in \ajf{can only produce?} more massive binaries, with total \ajf{black hole?} mass above $\sim 50 M_\odot$ \citep{MandeldeMink:2016,Marchant:2016}.

\subsubsection{Mass Ratios}
The mass ratios of merging black hole binaries produced through the isolated binary evolution channel are likely to be on the equal side of $2:1$, because of the mass ratio constraints necessary to ensure stable mass transfer from the primary to the secondary and then dynamically unstable reverse mass transfer. However, mass ratios exceeding $2:1$ are expected to be more common at low metallicity \citep{Dominik:2012,Stevenson:2017}.

Dynamically formed binaries may also favor comparable masses because the heaviest black holes are most likely to merge \ajf{say more here?}, though more extreme mass ratios are also possible if the globular cluster contains a particularly heavy stellar-mass black hole or an intermediate-mass black hole \citep{Mandel:2008,Belczynski:2014VMS}. 

Chemically homogeneous evolution has a much stronger preference for equal masses \citep{MandeldeMink:2016}, with perhaps nearly equal mass black holes resulting from the evolution of systems that went through a contact phase \ajf{what does "contact phase" mean for this channel?} \citep{Marchant:2016}.  

\subsubsection{Black Hole Spin Magnitudes and Directions}
The spins of merging black holes may carry imprints of their evolutionary history. 

 Isolated binaries (\ajf{and those formed by chemically homogeneous evolution?}) are generally expected to have preferentially aligned spins after undergoing episodes of mass transfer and/or tidal coupling, although natal kicks and possible spin tilts during supernovae could lead to misalignment \citep[e.g.,][]{Farr:2011}.  Meanwhile, the spin directions are likely to be isotropically distributed for dynamically formed binaries \citep[e.g.,][]{Rodriguez:2016spin}.

Spin magnitudes may also depend sensitively on evolutionary history.  Although a massive star may contain a lot of angular momentum, far in excess of the maximum value allowed for a spinning black hole, the vast majority of that angular momentum will be contained in the outer layers of the star, and can therefore be readily lost through winds or through envelope stripping by a companion.  Therefore, black holes formed from stars that were rapidly spinning at some point in their evolution may still spin slowly unless the stars are spun up through mass transfer or tides shortly before collapse \citep{Kushnir:2016,HotokezakaPiran:2017,Zaldarriaga:2017}.   This argument for slowly spinning black holes in compact binaries is potentially in conflict with the claimed observations of rapid spins in black-hole high mass X-ray binaries \citep{MillerMiller:2015}, but these \ajf{the observations, I assume?} may suffer from measurement systematics \citep[e.g.,][]{Kawano:2017}.  

Gravitational-wave spin measurements have so far only constrained the average spin component in the direction of the orbital angular momentum: the so-called `effective spin', which enters the phasing of gravitational waves at the 1.5 post-Newtonian order \citep{PoissonWill:1995}.  The low effective spins of the first four detected merging black holes allowed \citet{Farr:2017} to conclude that either the spin directions were isotropically distributed or the spin magnitudes were low.

\subsubsection{Orbital Eccentricities}
	The orbital eccentricities of merging black hole binaries may contain information about their past \citep{MandelOShaughnessy:2010}. Because gravitational waves are very efficient at damping out orbital eccentricity, both isolated binaries and those formed via the chemically homogeneous evolution channel are expected to merge on circular orbits. Dynamically formed systems may, however, retain significant eccentricities at merger if their orbits are very tight at formation. Dynamical formation of tight binaries may be possible either through two-body captures in Galactic nuclei \citep{OLeary:2008} \citep[but see][]{Tsang:2013}, or through close captures during three-body interactions \citep{Samsing:2014, Rodriguez:2018}. 


\subsubsection{Formation Environments}
\label{environ}
The evolutionary history of merging black holes is also dependent on their formation environments, particularly through the effect of chemical composition (metallicity) on stellar evolution and thereby black hole mass.

Black hole masses are set by the mass of the parent star at the end of stellar evolution, preceding the collapse or supernova explosion, and by the physics of the collapse itself. \textcolor{red}{--what does it mean to say that the mass is set by the physics?} Observational evidence and collapse models suggest that sufficiently massive stellar cores may completely collapse into black holes \citep[for a review, see][]{Mirabel:2016}. Thus, the mass of the stellar core -- the convective region in the center of the star where nuclear fusion proceeds from hydrogen through helium, carbon, oxygen, and so on to iron -- determines the ultimate black hole mass.  If left unperturbed, perhaps a third of the mass of a massive star will end up in a carbon--oxygen core, and thereby in the black hole remnant.

The intense radiation from massive stars -- whose luminosities may be a hundred thousand times greater than the Sun -- drives significant winds, which can remove much of the material from the star. The mass of the star at the end of stellar evolution can thus be much lower than the star's initial mass, and the mass of the resulting compact object can therefore be much lower than 1/3 of the initial stellar mass. Metals like iron, with their many absorption lines, are particularly efficient at capturing the stellar radiation and transforming it into outward momentum.  Therefore, the metal fraction, or metallicity -- the fraction of the star's mass that's in elements heavier than hydrogen and helium -- is a key ingredient in determining stellar wind mass loss rates \citep{Vink:2001}. Although wind models are uncertain \citep[e.g.,][]{Renzo:2017}, simulations suggest that at the 2\% metallicity of our solar neighborhood, massive stars lose enough mass via winds that maximum black hole masses are only around $15 M_\odot$ \citep{Belczynski:2009,Spera:2015}. This matches the masses of black holes observed in X-ray binaries. Figure \ref{fig:BHremnant} illustrates the dependence of compact remnant mass on metallicity. 
 
\begin{figure}
	\centering
	%\includegraphics[trim={1.3cm 7.0cm 0.0cm 7.9cm},clip,scale=0.45]{evolvedPvseWithOrbit.pdf}
	\includegraphics[width=0.7\textwidth]{BHremnantdelayed.png}
	\caption{\label{fig:BHremnant} Remnant compact object mass for a given zero-age main sequence stellar mass in the absence of an interacting binary companion, following the default COMPAS prescription for winds \citep{Stevenson:2017}  and the `delayed' supernova fallback prescription \citep{Fryer:2012}, at solar ($Z_\odot=0.02$) metallicity (blue) and a tenth solar (red).\ajf{distinguish between NSs and BHs?}} 
\end{figure}

Forming heavier black holes, like those observed through gravitational waves, requires stellar progenitors born in regions with lower metallicity. The yield of merging black holes per unit star-forming mass is also greatest at lowest metallicity, due largely to the reduced wind-driven mass loss \citep[e.g.,][]{Belczynski:2010}. For the isolated binary channel, delayed stellar expansion \ajf{at lower metallicity?}, which allows the stars to develop a larger helium core before engaging in mass transfer, further enhances this effect \citep[e.g.,][]{Stevenson:2017}.

The above considerations place constraints on the formation time of the systems that produced the observed merging black hole binaries. Most low-metallicity star formation occurs in the early Universe, before gas is polluted with metal-rich products of stellar evolution, although there is a broad range of metallicities at all times \citep[e.g.,][]{LangerNorman:2006,TaylorKobayashi:2015}. However, the delay time between star formation and merger in the classical formation channel peaks at shorter delay times with a $p(\tau) \propto 1/\tau$ decay. \ajf{need to say why -- does this just come out of population synthesis? Is it an approx power law?} A convolution of the metallicity-specific star formation history and the delay-time distribution led \citet{Belczynski:2016} to conclude that, a posteriori, there was a bimodal probability distribution on the formation time of the binary star progenitor of GW150914: it either formed in the first $\sim$ Gyr after the Big Bang, with a long subsequent delay to merger \citep{Dominik:2014}, or formed more recently in a low-metallicity region such as a low-mass satellite galaxy.  

  

\section{Prospects for gravitational-wave astronomy}\label{prospect}

\ajf{Might be worth reiterating at the start of this section that little can be said conclusively using the observations to date -- and then the next sentence about expected detection rates is more impactful -- we will know soon!}

Observed rates of binary black hole mergers indicate merger rates of around $\sim 10$ to $\sim 200$ Gpc$^{-3}$ yr$^{-1}$ \citep{GW150914:rates,GW170104}, which points to the prospect of tens of detections in the next observing run and hundred \ajf{hundreds or a hundred?} within the next three years, once the LIGO and Virgo instruments reach design sensitivity \citep{scenarios}.  This will yield enough information about both individually exciting events and population distributions to use gravitational-wave astronomy as a genuine tool for exploring stellar and binary evolution.  

\ajf{In this section we... describe some individually informative events and discuss some theoretical modeling techniques that will help to extract useful info from the observed data as the dataset grows in size... or something.}

\subsection{Individual exciting events}
Single events that would be individually informative include the discovery of a merging binary black hole with a component mass in the pair-instability supernova mass gap.  Models of pair instability supernovae suggest that no black holes with masses between $\sim 60 M_\odot$ (possibly reduced to $\sim 45 M_\odot$ by pulsational pair instability \citep{Woosley:2017}) and $\sim 130 M_\odot$ should form as a result of stellar collapse, with black hole formation possible on both sides of the mass gap \citep{Marchant:2016}.  Therefore, a discovery of a single object in this mass gap would point to either past dynamical mergers that left the merger remnant to be reused in a dense stellar environment, or to a theoretical failure in pair instability models. 

Similarly, the discovery of a merger component with a mass between $\sim 2.3 M_\odot$ and $\sim 5 M_\odot$, in the range between the most massive known neutron stars and least massive black holes \citep{Ozel:2010,Farr:2011} would shed light on supernova explosion models, specifically the amount of fallback that can occur before the explosion \citep{Fryer:2012}.

Other individually exciting events could include the first confirmed discovery of an intermediate mass black hole in the few-hundred solar-mass range, either as a merger of two such black holes \citep[e.g.,][]{Veitch:2015,Graff:2015} or as an intermediate-mass-ratio \ajf{not sure what this means} inspiral into such an object \citep[e.g.,][]{Haster:2015IMRI,Haster:2016}. 

A source with a detectable in-band \ajf{what band?} eccentricity would clearly signal a dynamical capture \citep{Breivik:2016}. 

A measurement of an effective spin value close to either 1 or $-1$ would point to rapid spins of both black holes. A substantial positive effective spin coupled with high masses would point to the likelihood of chemically homogeneous formation \citep{Marchant:2016}, while a substantially negative effective spin measurement would indicate the absence of alignment, likely either through a spin tilt or dynamical formation.  

\subsection{Population statistics}
Population statistics will generally yield more information than individual events. \ajf{needs another intro sentence explaining what you mean by population statistics, I think, before this jumps into pretty jargony talk, and we should also say what you're trying to do with the statistical analysis} We can separate the approaches to inference on the observed populations into two types: unmodeled or weakly modeled inference, and inference that relies on accurate models. 

At the most basic end of weakly modeled inference is the (possibly non-parametric) reconstruction of an underlying distribution, such as the mass function of merging black holes \citep{Mandel:2010stat,BBH:O1} while accounting for measurement uncertainties and selection effects.  \citet{Fishbach:2017mass} use a weakly parametrized model to argue for the possible evidence of a mass gap due to pair-instability supernovae, while \citet{TalbotThrane:2017} discuss weakly parametrized inference on the spin distributions.  Meanwhile, the time delay between star formation and binary merger could be extracted by comparing the merger rate as a function of redshift against the star formation rate as a function of redshift \citep{Mandel:2016select}.  

We can look for clustering of events by observed parameters in the hope of finding distinct clusters of systems corresponding to different subpopulations or evolutionary channels.  \citet{Mandel:2015} argued that $\sim 60$ observations would make it possible to cluster events by mass \ajf{why do you want to cluster them by mass?} even after allowing for the significant measurement uncertainties inherent in gravitational-wave astronomy, and \citet{Mandel:2016cluster} demonstrated the feasibility of a particular clustering technique.  Meanwhile, \citet{Farr:2018} propose classifying binary black hole mergers by spin misalignment angle into aligned and isotropically distributed subpopulations.  It is worth noting that such clustering or classification schemes cannot hope to correctly assign individual events to a specific cluster, which is generally impossible given the significant measurement uncertainties \citep{Littenberg:2015}; instead, the goal is to measure the relative frequencies of events in different categories.

The next level of population-based inference relies on assuming that precise, possibly parametrized, subpopulation distributions are known to determine the ratios of different subpopulations (e.g., arising from different formation channels), typically using hierarchical modeling \citep[extreme deconvolution in the language of][]{Hogg:2010}. \ajf{<--- this sounds super fancy but is also pretty impenetrable for me as a non-statistician} \citet{Zevin:2017} found that with $\sim 100$ observations, the mass distribution alone could be used to determine the formation channel, assuming the availability of trustworthy models of the mass distribution under different formation channels.  \citet{Vitale:2015,Stevenson:2017spin} carry out a similar investigation for spin-orbit misalignment angles; although these cannot be measured with the same accuracy as chirp masses, several hundred observations should again be sufficient to distinguish multiple formation channels through hierarchical modelling if the distributions under different channels are known \citep{Stevenson:2017spin}.  \ajf{I think the two examples here help explain what the start of the paragraph was about, but I still suggest trying to provide an initial explanation that's on a par with some of the astrophysics explanations in the article}

\subsection{Advanced population synthesis techniques}
There is, of course, a multitude of uncertain physics within any given model. \ajf{and this is a limitation for the above population statistics models?} How much mass do stars lose in winds, and what is the precise impact of metallicity and rotation on stellar evolution \citep[e.g.,][]{Renzo:2017}?  What happens to the star's angular momentum during collapse, how much mass is ejected, and how much of an asymmetric kick does the star receive \citep[e.g.,][]{Mirabel:2016}? How conservative is mass transfer in binaries and how much angular momentum is carried away by the mass lost from the binary during non-conservative mass transfer \citep[e.g.,][]{vandenHeuvel:2017}?  What are the conditions for the onset of a common-envelope phase and common envelope ejection, and how does the binary change in the process \citep[e.g.,][]{Ivanova:2013}?   How do dynamical interactions affect binary evolution?  And, in turn, how do massive stellar binaries and their compact remnants feed back into astrophysics and cosmology on all scales?

One approach to addressing these big questions is to use the framework of population synthesis, which makes it possible to parametrize the uncertain physics and predict the expected source rates and distributions under different models.  Early efforts \ajf{the references are to 2012 and 2015 articles -- this does not sound particularly early} relied on a discrete set of a few models in the parameter space of population synthesis assumptions \citep{Dominik:2012,Stevenson:2015}.  Even with the low computational cost of population synthesis, it is not feasible to explore more than a few tens to a few hundred models, which is not sufficient to explore the full parameter space or consider the correlations between model parameters.  However, recent successes in building accurate and computationally efficient emulators over the model parameter space \citep{Barrett:2017} suggest that a full exploration of this space will soon be possible. \citet{Barrett:2017FIM} applied Fisher information matrix techniques to the space of model parameters and found that parameters such as mass loss rates during the Wolf-Rayet phase and common envelope energetics can be measured to the level of a few percent with a thousand detections. 

\subsection{Other datasets and future missions}
Additional observational data sets will further aid in interpreting gravitational-wave observations and in refining models of binary stellar evolution.

Future gravitational-wave missions raise the prospect of observing the evolution of populations of merging binaries -- or individual systems -- across a broad band of frequencies, from the millihertz \citep[e.g.,][]{Sesana:2016} through the decihertz \citep{Mandel:2017} and hertz \citep{ET:2012}, to the LIGO/Virgo band.  \ajf{say briefly why this would be informative?}

Although electromagnetic transients are not broadly expected to be associated with binary black hole mergers \citep[e.g.,][]{Lyutikov:2016}, any such observations would indicate the persistence of material around the merging binary \citep[e.g.,][]{deMinkKing:2017}.  %\todo{More details -- brief summary of Lorentz center workshop?}

The most useful constraints are likely to come from the requirement that any candidate evolutionary model must self-consistently explain all of the available data: gravitational-wave observations, perhaps in multiple frequency bands, plus electromagnetic observations of X-ray binaries, Galactic neutron stars, gamma ray bursts, supernovae, luminous red novae, etc. Incorporating these constraints will make it possible to resolve modelling degeneracies and to build a concordance model of massive binary evolution.  %\todo{Other possible gravitational-wave sources? More details / references on XRBs, LRN, etc.?}

\subsection{Outlook}

\ajf{Tie things together here before the dinosaur bit. Right now, we have a few observations and some plausible evolutionary channels. It's not possible to tell them apart yet statistically, but we have the tools in place to do so and we expect to have enough data to do so within the next few years...? }

\ajf{Also: is the field generally leaning in a particular direction at the moment? Initial suspicions? And what would be surprising?}

The future of gravitational-wave astronomy holds the prospect of addressing the inverse problem of massive binary stellar evolution: inferring the formation channels and their physics from observations of the merging compact-object binary population.  Like a paleontologist who uses her knowledge of anatomy to determine the appearance, eating habits, and even behaviour of extinct dinosaurs from their fossilized remnants \ajf{love it ;) }, we can now use merging black holes --- remnants of massive stars --- to probe the behaviour of those stars, and particularly their evolution in binaries \citep{MandelFarmer:2017}.   With the data to come and the theoretical developments underway, this future is brightly chirping indeed! \ajf{hmm... needs a better last sentence or two, or we need to have mentioned chirping more previously...!}


\begin{acknowledgements}
IM thanks all of the people who have guided him to understand stellar and binary evolution, foremost among them Selma de Mink.   The students who contributed to building COMPAS (Simon Stevenson, Alejandro Vigna G\'{o}mez, Coenraad Neijssel -- who deserves special thanks for assistance with figures \ref{fig:BHremnant} and \ref{fig:Rmax}, Jim Barrett, Sebastian Gaebel) and external collaborators (including Stephen Justham and Philipp Podsiadlowski) deserve particular thanks.  Thanks also to Cole Miller, Vicky Kalogera, and Will Farr for stimulating discussions spanning many years. 

AF is grateful for the astro-tourism.
\end{acknowledgements}

\bibliographystyle{hapj}
\bibliography{Mandel}

\end{document}

